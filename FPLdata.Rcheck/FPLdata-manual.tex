\nonstopmode{}
\documentclass[a4paper]{book}
\usepackage[times,inconsolata,hyper]{Rd}
\usepackage{makeidx}
\usepackage[utf8]{inputenc} % @SET ENCODING@
% \usepackage{graphicx} % @USE GRAPHICX@
\makeindex{}
\begin{document}
\chapter*{}
\begin{center}
{\textbf{\huge Package `FPLdata'}}
\par\bigskip{\large \today}
\end{center}
\inputencoding{utf8}
\ifthenelse{\boolean{Rd@use@hyper}}{\hypersetup{pdftitle = {FPLdata: Read in Fantasy Premier League Data}}}{}\begin{description}
\raggedright{}
\item[Title]\AsIs{Read in Fantasy Premier League Data}
\item[Version]\AsIs{0.1.0}
\item[Author]\AsIs{person(given = ``Andrew'',
family = ``Little'',
role = c(``aut'', ``cre''),
email = ``andrewlittlebristol@gmail.com'')}
\item[Maintainer]\AsIs{Andrew Little }\email{andrewlittlebristol@gmail.com}\AsIs{}
\item[Description]\AsIs{This data contains a large variety of information on players and their
current attributes on Fantasy Premier League
<}\url{https://fantasy.premierleague.com/}\AsIs{>. In particular, it contains a
\textbackslash{}code\{next_gw_points\} (next gameweek points) value for each player
given their attributes in the current week. Rows represent player-gameweeks,
i.e. for each player there is a row for each gameweek. This
makes the data suitable for modelling a player's next gameweek points, given
attributes such as form, total points, and cost at the current gameweek.
This data can therefore be used to create Fantasy Premier League bots that
may use a machine learning algorithm and a linear programming solver
(for example) to return the best possible transfers and team to pick for
each gameweek, thereby fully automating the decision making process in
Fantasy Premier League. This function simply supplies the required data
for such a task.}
\item[Imports]\AsIs{readr, dplyr}
\item[License]\AsIs{MIT + file LICENSE}
\item[Encoding]\AsIs{UTF-8}
\item[LazyData]\AsIs{true}
\item[RoxygenNote]\AsIs{7.1.2}
\end{description}
\Rdcontents{\R{} topics documented:}
\inputencoding{utf8}
\HeaderA{FPLdata}{Read in Fantasy Premier League Data}{FPLdata}
%
\begin{Description}\relax
Read in the weekly-updated Fantasy Premier League football data from the
GitHub repository <https://github.com/andrewl776/fplmodels>.
\end{Description}
%
\begin{Usage}
\begin{verbatim}
FPLdata()
\end{verbatim}
\end{Usage}
%
\begin{Details}\relax
This data contains a large variety of information on players and their
current attributes on Fantasy Premier League
<https://fantasy.premierleague.com/>. In particular, it contains a
\code{next\_gw\_points} (next gameweek points) value for each player
given their attributes in the current week. Rows represent player-gameweeks,
i.e. for each player there is a row for each gameweek. This
makes the data suitable for modelling a player's next gameweek points, given
attributes such as form, total points, and cost at the current gameweek.
This data can therefore be used to create Fantasy Premier League bots that
may use a machine learning algorithm and a linear programming solver
(for example) to return the best possible transfers and team to pick for
each gameweek, thereby fully automating the decision making process in
Fantasy Premier League. This function simply supplies the required data
for such a task.
\end{Details}
%
\begin{Value}
A dataframe (tibble).
\end{Value}
%
\begin{Examples}
\begin{ExampleCode}
{

library(dplyr)

fpl_data <- FPLdata()

head(fpl_data)

fpl_data %>%
  group_by(web_name) %>%
  summarise("mean_next_gw_points" = mean(next_gw_points, na.rm = TRUE)) %>%
  arrange(-mean_next_gw_points)

}
\end{ExampleCode}
\end{Examples}
\printindex{}
\end{document}
